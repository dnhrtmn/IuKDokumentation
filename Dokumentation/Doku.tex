\documentclass[a4paper,report,headsepline]{scrreprt} 

\usepackage{xcolor} 
\usepackage{graphicx}
\usepackage[german]{babel}
\usepackage[utf8]{inputenc}
\usepackage[T1]{fontenc}
\usepackage{ae}
\usepackage{comment}



\usepackage[automark]{scrlayer-scrpage} 

% headsepline=Dicke:Länge (jeweils optional, die Dicke ist auf 0.4pt voreingestellt)
\KOMAoptions{headsepline=0.5pt} 
\addtokomafont{headsepline}{\color{black}}

\begin{document}


\title{ \begin{Huge}
\textbf{Projektdokumentation}
\end{Huge} \\  \dq Learning Spaces\dq}
\author{Anika Balke, Daniel Hartmann, Lisa Peter, \\ Maximilian Hartwich, Yuliya Karybochkin}
%\date{} %%Wenn kommentiert, wird das aktuelle Datum verwendet.

\maketitle


\tableofcontents
\clearpage



\chapter{Meeting Protokolle}
\underline{{\large Mittwoch, 06.05.2020 Beginn: 10:00 Uhr Ende 10:30 Uhr}}  \\
\textit{Teilnehmer: Lisa Peter, Anika Balke, Maximilian Hartwich, Daniel Hartmann, Yuliya Karyabochkina}

 \begin{itemize}
  \item Allgemeine Besprechung des Vorgehens bei dem Projekt, grobe Zeitplanung der nächsten Wochen
 \item Analyse des Lastenheftes
 \item Verteilung der Aufgaben
 
 \begin{itemize}
 \item Daniel: Festlegung als Product Owner
 \item Anika, Lisa und Yuliya: Erstellung des Pflichtenheftes
 \item Maximilian: Erstellung des UML-Diagrammes sowie der Aktivi-
 tätendiagramme
 \item Yuliya: Erstellung des Use-Case Diagrammes
 \item Daniel: Erste Schritte der Codierung (ausschließlich bekannte Musskriterien des Lastenheftes)
 
\end{itemize} 
\end{itemize}  
 \underline{{\large Mittwoch, 13.05.2020 Beginn: 20:00 Uhr Ende: 20:30 Uhr}}  \\
\textit{Teilnehmer: Lisa Peter, Anika Balke, Maximilian Hartwich, Daniel Hartmann}

\begin{itemize}
 \item Absprache der bisherigen Ergebnisse
\item Sammlung von Fragen für Termin am nächsten Tag mit Herrn Schmidt:
\begin{itemize}
 \item Nur eine Buchung pro Woche: Ist nach Stornierung eine Buchung für die gleiche Woche möglich oder erst für die kommende Woche?
\item Wie detailliert sollen die Aktivitäten im Aktivitätsdiagramm beschrieben werden?
\item Wie detailliert sollen die Aktivitäten im Use-Case Diagramm beschrieben werden?
\item Welche Räume sollen installiert werden? (Größe, Lage, usw.)
\item Ist ein genauer Platz in den Räumen buchbar oder nur allgemein ein Platz?
\item Blocklänge: mit (Mittags-) Pause?
\item Dürfen die Belegungen der Räume öffentlich einsehbar sein oder nur für eingeloggte User? 

\end{itemize} 
\end{itemize}  
 \underline{{\large Donnerstag, 14.05.2020 Beginn: 9:00 Uhr Ende: 9:45 Uhr}}  \\
\textit{Teilnehmer: Lisa Peter, Anika Balke, Maximilian Hartwich, Daniel Hartmann, Yuliya Karyabochkina, Michael B. Schmidt}

\begin{itemize}
\item Allgemeine Besprechung des Projektstatus
\item Kundengespräch mit folgenden Festlegungen und Inhalten: 
\begin{itemize}
\item Ansicht der Belegungen maximal eine Woche im Voraus
\item Jeden Vorgang mit Hilfe eines Aktivitätendiagrammes beschreiben
\item Größe der Räume soll von der Gruppe definiert werden
\item Raumbuchungen dürfen nur für eingeloggte User sichtbar sein
\item Bestätigungsemail bei erfolgreicher Raumbuchung als Wunschkriterium

\end{itemize} 
\end{itemize}
 \underline{{\large Mittwoch, 20.05.2020, Beginn: 10:00 Uhr Ende: 10:45 Uhr}}  \\
\textit{Teilnehmer: Lisa Peter, Anika Balke, Maximilian Hartwich, Daniel Hartmann, Yuliya Karyabochkina}

\begin{itemize}
\item Gemeinsame Besprechung:

\begin{itemize}
\item Bisher angefertigter Diagramme
\item Aktueller Stand des Pflichtenheftes
\end{itemize}

\item Klärung verschiedener Fragen zu den einzelnen Funktionen für die Erstellung des Pflichtenheftes
\item Besprechung des aktuellen Programmierstandes und Festlegung des Programmierstandes bis zum nächsten Kundentermin mit Herrn Schmidt:
\begin{itemize}
\item Layout der Startseite
\item Buchungsübersicht
\item Layout bei dem Buchungsvorgang

\end{itemize}
\end{itemize} 
 \underline{{\large Dienstag, 26.05.2020, Beginn: 14:15 Uhr, Ende 15:15 Uhr}}  \\
\textit{Teilnehmer: Lisa Peter, Anika Balke, Maximilian Hartwich, Daniel Hartmann, Yuliya Karyabochkina, Michael B. Schmidt}

\begin{itemize}
\item Allgemeine Besprechung des Projektstatus
\item Besprechung verschiedener Funktionen
\item Weitere Muss- und Wunschkriterien mit Herrn Schmidt abgesprochen
\item Diverse Rahmenbedingungen abgeklärt
\item Abgleich der Diagramme mit Herrn Schmidt
\item Präsentation einer bisherigen Layoutsskizze 

\end{itemize}
 \underline{{\large 04.06.2020 Beginn: 19:50 Uhr Ende: 21:30 Uhr}}  \\
\textit{Teilnehmer: Lisa Peter, Anika Balke, Maximilian Hartwich, Daniel Hartmann, Yuliya Karyabochkina}

\begin{itemize}
\item Durchsprache der bisherigen Ergebnisse aller Gruppenmitglieder
\item Besprechung der Ergebnisse im Pflichtenheft, Einfügen der erforderlichen Diagramme (UML-Diagramm, Use-Case Diagramm, Aktivitätendiagramme)
\item Anpassung und Überarbeitung des Pflichtenheftes

\end{itemize}

 
\section{Scrub-Meetings} 
 \underline{{\large SCRUM-Meeting Nr. 1, Mittwoch, 10.06.2020, Beginn: 10:00 Uhr Ende 10:30 Uhr}}  \\
\textit{Teilnehmer: Lisa Peter, Anika Balke, Maximilian Hartwich, Daniel Hartmann, Yuliya Karyabochkina}

\begin{itemize}
\item Festlegen der Sprintdauer auf eine Woche
\item Verteilung der eingehenden Aufgaben:
\begin{itemize}
\item Nach Feedback des Kundens (26.05.2020) detailliertere Codierung der bisherigen Benutzeroberfläche (Startseite, Buchungsübersicht, Buchungsvorgang) --> Daniel (2 Sprints)
\item Erstellung der Datei für die Dokumentation und Bereitstellung auf GitHub $\rightarrow$ Lisa \& Anika (1 Sprint)
\item Planung der Implementierung der Funktionen in die Seite (1 Sprint) --> Maximilian \& Yuliya 
\end{itemize}
\end{itemize}
 \underline{{\large SCRUM-Meeting Nr. 2, Mittwoch, 17.06.2020, Beginn: 10:30 Uhr Ende 11:15 Uhr}}  \\
\textit{Teilnehmer: Lisa Peter, Anika Balke, Maximilian Hartwich, Daniel Hartmann, Yuliya Karyabochkina}
      

\begin{itemize}
\item Durchsprache der bisherigen Ergebnisse
\item Klärung offener Fragen (Probleme bei der Dokumentation auf GitHub)
\item Verteilung der Aufgaben für die kommende Woche
\begin{itemize}
\item Anmeldung programmieren: Eingabefelder für Benutzername und Passwort, Login-Button, Design der Startseite --> Daniel (1 Sprint übrig)
\item Coding der Raumübersicht: Ansicht verschiedener Räume mit Infotext über jeweiligen Raum --> Yuliya \& Maximilian (2 Sprints)
\item Problembehebung auf GitHub --> Lisa \& Anika (1 Sprint)

\end{itemize}
\end{itemize}
 \underline{{\large SCRUM-Meeting Nr. 3, Mittwoch, 24.06.2020, Beginn: 10:30 Uhr Ende 11:15 Uhr}}  \\
\textit{Teilnehmer: Lisa Peter, Anika Balke, Maximilian Hartwich, Daniel Hartmann, Yuliya Karyabochkina}      

\begin{itemize}
\item Durchsprache aktueller Stand, Probleme bei der Programmierung: es wurde festgestellt, dass kleinere Pakete verteilt werden müssen (Statt „Codierung Startseite“ eher „Codierung Login-Button“ und „Codierung Eingabefelder“
\item Aufgabenverteilung für die nächste Woche:
\begin{itemize}
\item Coding der zur Verfügung stehenden Zeitslots --> Anika (1 Sprint)
\item Coding: maximal eine Buchung pro Woche möglich --> Lisa (1 Sprint)
\item Coding der Reservierungsübersicht: Ansicht eines jeweiligen Raumes mit allen notwendigen Angaben (Datum, Raum, Zeitblock), Button zur Stornierung der angezeigten Reservierung --> Maximilian (2 Sprints)
\item PopUp zur Bestätigung der Reservierung --> Yuliya (1 Sprint)

\end{itemize}
\end{itemize}
 \underline{{\large SCRUM-Meeting Nr. 4, Mittwoch, 01.07.2020, Beginn: 17:45 Uhr Ende 18:15 Uhr}}  \\
\textit{Teilnehmer: Lisa Peter, Anika Balke, Maximilian Hartwich, Daniel Hartmann, Yuliya Karyabochkina} 

\begin{itemize}
\item Rücksprache über aktuelle Arbeitspakete (erledigt, Probleme, Verlängerung notwendig oder Sonstiges)
\item Aufgabenverteilung für die nächste Woche:
\begin{itemize}
\item PopUp bei belegtem Space und User --> Anika (1 Sprint)
\item Überbuchungsfunktion, wenn User = Mitarbeiter --> Daniel (1 Sprint)
\item Eigene Raumbuchung anzeigen lassen --> Yuliya (1 Sprint)
\item Stornierungsfunktion --> Maximilian (1 Sprint)
\item Anzeigeoption freier Räume über Suchfunktion --> Daniel (1 Sprint)
\item Anzeige zur Verfügung stehender Funktionen --> Lisa (1 Sprint)

\end{itemize}
\end{itemize}
 \underline{{\large SCRUM-Meeting Nr. 5, Mittwoch, 08.07.2020, Beginn: 12:30 Uhr Ende 13:30 Uhr}}  \\
\textit{Teilnehmer: Lisa Peter, Anika Balke, Maximilian Hartwich, Daniel Hartmann, Yuliya Karyabochkina} 

\begin{itemize}
\item Allgemeine Besprechung des Projektstatus
\item Rücksprache über aktuelle Arbeitspakete (erledigt, Probleme, Verlängerung notwendig oder Sonstiges)
\item Aufgabenverteilung für die nächste Woche:
\begin{itemize}
\item Hilfe über Kontaktformular anfragen --> Maximilian (1 Sprint)
\item E-Mail zur Benachrichtigung bei Buchung, Stornierung --> Daniel (1 Sprint)
\item Internationalisierung des Systems (System in englischer Sprache) 
--> Yuliya (1 Sprint)
\item Abschlussdokumentation der Projektarbeit --> Anika \& Lisa

\end{itemize}
\end{itemize}
 \underline{{\large SCRUM-Meeting Nr. 6, Mittwoch, 15.07.2020, Beginn: 10:30 Uhr Ende 11:15 Uhr}}  \\
\textit{Teilnehmer: Lisa Peter, Anika Balke, Maximilian Hartwich, Daniel Hartmann, Yuliya Karyabochkina}

\begin{itemize}
\item Rücksprache über aktuelle Arbeitspakete (erledigt, Probleme, Verlängerung notwendig oder Sonstiges)
\item Sammeln von Ideen für mögliche weitere „Kann“-Funktionen und Aufteilung einiger auf die nächste Woche
\begin{itemize}
\item Anonyme Anfrage belegter Räume --> Daniel (1 Sprint)
\item Hinweis, bei falscher Kennworteingabe beim Login --> Maximilian (1 Sprint)
\item Kalenderfunktion --> Yuliya (1 Sprint)
\item Anzeige personenbezogener Daten --> Daniel (1 Sprint)
\item Abschlussdokumentation in Latex --> Lisa \& Anika (Running ToDo bis zum Projektabschluss)

\end{itemize}
\end{itemize}
 \underline{{\large SCRUM-Meeting Nr. 7, Mittwoch, 22.07.2020, Beginn: 10:30 Uhr Ende 11:15 Uhr}}  \\
\textit{Teilnehmer: Lisa Peter, Anika Balke, Maximilian Hartwich, Daniel Hartmann, Yuliya Karyabochkina}

\begin{itemize}
\item Rücksprache über aktuelle Arbeitspakete (erledigt, Probleme, Verlängerung notwendig oder Sonstiges)
\item Aufgabenverteilung für die nächste Woche:
\begin{itemize}
\item Änderung des eigenen Passworts / der E-Mail Adresse --> Daniel (1 Sprint)
\item Logout --> Yuliya (1 Sprint)
\item Anlegen, Editieren und Löschen von Benutzerkonten durch den Admin --> Maximilian (1 Sprint)
\item Anlegen und Editieren von Space-bezogenen Daten ausschließlich durch den Admin --> Maximilian (1 Sprint)
\item Anlegen, Editieren und Löschen der Ressourcen der Learning Spaces --> Daniel (2 Sprints)
\item Abschlussdokumentation in Latex --> Lisa \& Anika (Running ToDo bis zum Projektabschluss)

\end{itemize}
\end{itemize}
 \underline{{\large SCRUM-Meeting Nr. 8, Mittwoch, 29.07.2020, Beginn: 10:15 Uhr Ende 11:45 Uhr}}  \\
\textit{Teilnehmer: Lisa Peter, Anika Balke, Maximilian Hartwich, Daniel Hartmann, Yuliya Karyabochkina}
\begin{itemize}
\item Allgemeine Besprechung des Projektstatus
\item Rücksprache über aktuelle Arbeitspakete (erledigt, Probleme, Verlängerung notwendig oder Sonstiges)
\item Sammeln erster Ideen für mögliche Testszenarien --> alle
\item Einfügen fehlender Kommentare im Quelltext --> Daniel \& Maximilian (Running ToDo bis zum Projektabschluss)
\item Fertigstellung der Programmierung, damit ab nächster Woche Testszenarien durchgeführt werden können und Fehlermeldungen behoben werden können. --> Daniel \& Maximilian (Running ToDo bis zum Projektabschluss)
\item Aufgabenverteilung für die nächste Woche:
\begin{itemize}
\item Abschlussdokumentation der Projektarbeit --> Anika \& Lisa
\item Beschreibung der API Endpunkte --> Daniel (2 Sprints)

\end{itemize}
\end{itemize}
 \underline{{\large SCRUM-Meeting Nr. . 9, Mittwoch, 05.08.2020, Beginn: 10:00 Uhr Ende 11:30 Uhr}}  \\
\textit{Teilnehmer: Lisa Peter, Anika Balke, Maximilian Hartwich, Daniel Hartmann, Yuliya Karyabochkina}
\begin{itemize}
\item Durchsprache des aktuellen Standes des Projektes
\item Besprechung des Zeitplans / noch durchzuführender Aufgaben bis Ende August
\item Durchführung eines ersten Testszenarios - Student: 
\begin{itemize}
\item Anzeige aller möglicher zu buchender Learning Spaces
\item Buchung eines Learning Spaces zu einem bestimmten Block
\item Anzeige aller persönlichen getätigten Buchungen
\item Stornierung der getätigten Buchung
\item Über Kontaktformular automatische Kontaktaufnahme bei Nachfragen / Komplikationen
\end{itemize}
\item Aufgabenverteilung für die nächste Woche:
\begin{itemize}
\item Abschlussdokumentation der Projektarbeit --> Anika \& Lisa
\item Beschreibung der API Endpunkte --> Yuliya (1 Sprint übrig)
\item Beschreibung des durchgeführten Tests --> Anika \& Lisa (1Sprint)
\item Einfügen fehlender Kommentare im Quelltext --> Daniel \& Maximilian (Running ToDo bis zum Projektabschluss)
\item Fertigstellung der Programmierung, damit ab nächster Woche Testszenarien durchgeführt werden können und Fehlermeldungen behoben werden können. --> Daniel \& Maximilian (Running ToDo bis zum Projektabschluss)

\end{itemize}
\end{itemize}
 \underline{{\large SCRUM-Meeting Nr. 10, Mittwoch, 12.08.2020, Beginn: 09:00 Uhr Ende 10:45 Uhr}}  \\
\textit{Teilnehmer: Lisa Peter, Anika Balke, Maximilian Hartwich, Daniel Hartmann, Yuliya Karyabochkina}
\begin{itemize}
\item Allgemeine Besprechung des Projektstatus
\item Durchführung eines ersten Testszenarios - Dozent: 
\begin{itemize}
\item Anzeige aller möglicher zu buchender Learning Spaces
\item Buchung eines Learning Spaces zu einem bestimmten Block
\item Anzeige aller persönlichen getätigten Buchungen
\item Stornierung der getätigten Buchung
\item Überbuchung eines bereits belegten Learning Spaces zu einem bestimmten Zeitpunkt
\end{itemize}
\item Aufgabenverteilung für die nächste Woche
\begin{itemize}
\item Abschlussdokumentation der Projektarbeit --> Anika \& Lisa
\item Beschreibung des durchgeführten Tests --> Yuliya (1 Sprint)
\item Durchführung letzter Feinheiten an der Programmierung --> Daniel \& Maximilian (1Sprint)

\end{itemize}
\end{itemize}

      
     
\chapter{Github Repo}
\chapter{Fortschreibung des Zeitplans mit den erledigten Arbeitspaketen}
\chapter{Produktdokumentation basierend auf dem Pflichtenheft}
\chapter{Beschreibung der durchgeführten Tests}
\chapter{Beschreibung der API Endpunkte}        

\end{document}